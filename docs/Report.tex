\documentclass[11pt, a4paper]{article}

% --- STANDARD PREAMBLE FOR PDFLATEX ---
\usepackage[utf8]{inputenc}
\usepackage[T1]{fontenc}
\usepackage{lmodern} % Latin Modern (serif) fonts
\usepackage[english]{babel}
\usepackage[a4paper, top=2.5cm, bottom=2.5cm, left=2cm, right=2cm]{geometry}

% Packages for math and formatting
\usepackage{amsmath}
\usepackage{amssymb}
\usepackage{booktabs}
\usepackage{graphicx}

\usepackage{hyperref}
\usepackage{float}
\usepackage{enumitem}
\usepackage{parskip}
\usepackage{listings}
\usepackage{xcolor}

% Hyperlink setup
\hypersetup{
    colorlinks=true,
    linkcolor=blue,
    filecolor=magenta,
    urlcolor=blue,
    pdftitle={Baswana-Sen Spanner Project},
    pdfpagemode=FullScreen
}

% Code listing style
\definecolor{codegreen}{rgb}{0,0.6,0}
\definecolor{codegray}{rgb}{0.5,0.5,0.5}
\definecolor{codepurple}{rgb}{0.58,0,0.82}
\definecolor{backcolour}{rgb}{0.95,0.95,0.92}

\lstdefinestyle{mystyle}{
    backgroundcolor=\color{backcolour},
    commentstyle=\color{codegreen},
    keywordstyle=\color{magenta},
    numberstyle=\tiny\color{codegray},
    stringstyle=\color{codepurple},
    basicstyle=\ttfamily\footnotesize,
    breakatwhitespace=false,
    breaklines=true,
    captionpos=b,
    keepspaces=true,
    numbers=left,
    numbersep=5pt,
    showspaces=false,
    showstringspaces=false,
    showtabs=false,
    tabsize=2
}
\lstset{style=mystyle}

% Title Page Setup (same font as body text: NO \sffamily)
\title{
    \vspace{2cm}
    \textbf{\Huge Mini-Project on Low-Distortion Embeddings:\\[0.5em]
    The Baswana-Sen Algorithm for Sparse Spanners}
}
\author{
    \Large \textbf{Tamir Cohen} \\[0.5em]
    \large Course: 202-1-4071 (Prof. Michael Elkin)
}
\date{\large December 2025}

\begin{document}

\maketitle
\thispagestyle{empty} % No page number on title page

\begin{center}
    \large \href{https://github.com/Tamircohen28/baswana-sen-spanner-experiments}{Project Repository}
\end{center}

\begin{abstract}
\noindent
This report presents a comprehensive study, implementation, and experimental analysis of the Baswana-Sen randomized algorithm for constructing $(2k-1)$-spanners in weighted graphs. Graph spanners are sparse subgraphs that approximate pairwise distances within a guaranteed distortion factor (stretch). We implemented the algorithm in Python and conducted a systematic evaluation on Erd\H{o}s--R\'{e}nyi graphs with sizes $n \in \{500, 1000, 2000\}$. Our experiments, covering 540 unique configurations, empirically validate that the algorithm achieves the expected size $O(k \cdot n^{1+1/k})$ and satisfies the $(2k-1)$ stretch bound. Furthermore, we demonstrate that the average stretch is significantly lower than the worst-case bound, and the algorithm exhibits robust performance across varying graph densities. The experiments were conducted on state-of-the-art hardware (Apple M4 chip), taking approximately 1.5 hours to complete.
\end{abstract}

\clearpage
\setcounter{page}{1} % Start page numbering here

% Requirement 1: The paper from which the alg' is taken
\section{The Paper and Algorithm Selection}

\subsection*{Selected Paper}
\textbf{Baswana, S., \& Sen, S. (2007).} "A Simple and Linear-Time Randomized Algorithm for Computing Sparse Spanners in Weighted Graphs." \textit{Random Structures \& Algorithms}, 30(4), 532--563. (Originally presented in ICALP 2003).

\subsection*{Context}
The problem of computing a $t$-spanner involves finding a subgraph $H = (V, E_S)$ of a graph $G = (V, E)$ such that the number of edges $|E_S|$ is minimized, while satisfying the distance constraint for all vertex pairs:
\[
d_H(u, v) \le t \cdot d_G(u, v)
\]
where $t$ is the ``stretch factor.''

\subsection*{Motivation}
Prior to Baswana and Sen's work, algorithms for constructing $(2k-1)$-spanners typically relied on constructing shortest-path trees or BFS trees, often incurring high time complexity ($O(m \cdot n^{1/k})$ or worse). The Baswana-Sen algorithm is a landmark contribution because it introduces a \textbf{linear-time} approach ($O(km)$ expected time) that relies purely on local clustering and random sampling, avoiding costly global distance computations entirely. It achieves a size of $O(k n^{1+1/k})$, matching the theoretical lower bound based on the Erd\H{o}s Girth Conjecture.

\clearpage

% Requirement 2: The alg' itself
\section{The Baswana-Sen Algorithm}

The algorithm is a randomized Las Vegas algorithm that constructs a $(2k-1)$-spanner. It operates in $k$ phases to iteratively cluster vertices and select edges that preserve connectivity and distance bounds.

\subsection{Algorithmic Phases}
The algorithm proceeds in phases indexed $i=1$ to $k$:

\begin{itemize}
    \item \textbf{Initialization:} The initial clustering $\mathcal{C}_0$ consists of $n$ singleton clusters $\{ \{v\} \mid v \in V \}$.

    \item \textbf{Phases $1$ to $k-1$ (Clustering):}
    \begin{enumerate}
        \item \textbf{Sampling:} A set of clusters $\mathcal{R}_i$ is sampled from the previous clustering $\mathcal{C}_{i-1}$. Each cluster is sampled independently with probability $n^{-1/k}$.
        \item \textbf{Connection:} Vertices that are not part of sampled clusters connect to their nearest neighboring sampled cluster via a least-weight edge. This forms the new clustering $\mathcal{C}_i$.
        \item \textbf{Spanner Edges:} Edges connecting vertices to their cluster centers are added to the spanner $H$. Crucially, for vertices that \textit{cannot} join a sampled cluster (because they have no neighbors in $\mathcal{R}_i$), we add the least-weight edge to \textit{every} neighboring cluster to $H$.
    \end{enumerate}

    \item \textbf{Phase $k$ (Final Cleanup):} The sampling stops. For every vertex $v$ and every neighboring cluster $c$ it is adjacent to (where $v \notin c$), we add the least-weight edge between $v$ and $c$ to $H$. This ensures global connectivity and satisfies the stretch requirement.
\end{itemize}

\subsection{Complexity Analysis}
\begin{itemize}
    \item \textbf{Time:} With $k$ phases, the total time is expected $O(km)$.
    \item \textbf{Size:} The probabilistic analysis guarantees an expected size of $O(k n^{1+1/k})$.
    \item \textbf{Stretch:} The construction ensures that if an edge $(u,v)$ is discarded, there exists a path in $H$ between $u$ and $v$ of weight at most $(2k-1) \cdot w(u,v)$.
\end{itemize}

\clearpage

% Requirement 3: Your implementation of the alg'
\section{Implementation of the Algorithm}

We implemented the Baswana-Sen algorithm in Python, prioritizing modularity and reproducibility. The complete source code is available in the project repository.

\subsection{Code Structure}
The implementation is organized as a Python package \texttt{baswana-sen-spanner-experiments}. Key files include:

\begin{itemize}
    \item \textbf{\href{https://github.com/Tamircohen28/baswana-sen-spanner-experiments/blob/main/src/spanners/baswana_sen.py}{src/spanners/baswana\_sen.py}}: Contains the core logic in the function \texttt{build\_spanner\_baswana\_sen}.
    \item \textbf{\href{https://github.com/Tamircohen28/baswana-sen-spanner-experiments/blob/main/src/graphs/erdos_renyi.py}{src/graphs/erdos\_renyi.py}}: Utilities for generating random graphs and extracting the largest connected component.
    \item \textbf{\href{https://github.com/Tamircohen28/baswana-sen-spanner-experiments/blob/main/src/utils/seeding.py}{src/utils/seeding.py}}: Ensures deterministic seeding for reproducibility.
\end{itemize}

\subsection{Key Implementation Details}
\begin{itemize}
    \item \textbf{Graph Representation:} We utilized Python dictionaries (\texttt{Dict[int, List[int]]}) for adjacency lists. This allows for $O(1)$ edge lookups and avoids the overhead of heavy graph libraries like NetworkX during the critical inner loops of the algorithm.
    \item \textbf{Randomized Clustering:} The clustering probability is calculated as \texttt{prob = n ** (-1.0 / k)}. We use a unified seeding mechanism to set both Python's \texttt{random} and \texttt{numpy.random} seeds, ensuring that every experiment is fully reproducible.
    \item \textbf{Connectivity Assurance:} The algorithm requires a connected graph. Our generator creates an Erd\H{o}s--R\'{e}nyi graph $G(n,p)$ and explicitly extracts the \textbf{largest connected component} (LCC) using BFS before passing it to the spanner algorithm.
\end{itemize}

\clearpage

% Requirement 4: The experiments you chose to run
\section{Experimental Design}

We designed a comprehensive experiment suite to validate the algorithm's performance across different scales and densities.

\subsection{Parameters}
\begin{itemize}
    \item \textbf{Graph Sizes ($n$):} We tested $n \in \{500, 1000, 2000\}$.
    \item \textbf{Stretch Parameter ($k$):} We tested $k \in \{2, 3, 4, 5\}$, corresponding to theoretical max stretches of $3, 5, 7, 9$.
    \item \textbf{Edge Probability ($p$):} We tested 9 unique densities per $n$ value to cover various regimes:
    \begin{itemize}
        \item \textbf{Sparse:} $p \approx \frac{\log n}{n}$ (near connectivity threshold).
        \item \textbf{Medium:} $p \approx n^{-0.5}$.
        \item \textbf{Dense:} $p \in \{0.1, 0.2, 0.3\}$.
    \end{itemize}
    \item \textbf{Repetitions:} Each configuration $(n, p, k)$ was run \textbf{5 times} with different deterministic random seeds.
    \item \textbf{Sampling:} For efficiency, stretch was estimated using \textbf{1,000 sampled edges} and \textbf{1,000 sampled vertex pairs} per experiment.
\end{itemize}

\subsection{Metrics}
For each trial, we recorded:
\begin{enumerate}
    \item \textbf{Sparsity:} The number of edges $|E(H)|$ and the \textbf{Spanner Size Ratio}: $\frac{|E(H)|}{k \cdot n^{1+1/k}}$.
    \item \textbf{Max Stretch:} The maximum ratio $\delta_H(u,v) / \delta_G(u,v)$ over sampled edges.
    \item \textbf{Average Stretch:} The mean stretch computed over sampled edges and vertex pairs.
    \item \textbf{Runtime:} Breakdown of time spent on Graph Generation vs. Spanner Construction.
\end{enumerate}

\clearpage

% Requirement 5: The implementation of those experiments
\section{Implementation of Experiments}

The experiments were executed using the command-line script \texttt{scripts/run\_all\_experiments.py}. This script orchestrates the generation, construction, and evaluation phases, saving results incrementally to timestamped CSV files.

\subsection{Hardware and Constraints}
The experimental suite was executed on a machine equipped with an \textbf{Apple M4 chip}, which represents state-of-the-art consumer hardware. The total execution time was approximately \textbf{1.5 hours}.

Despite the high performance of the M4 chip, we capped our experiments at $n=2000$ due to time constraints. While we identified potential performance optimizations, such as using multi-processing to parallelize the 540 trials, we encountered technical stability issues during implementation. Given the strict deadlines, we opted for a robust single-threaded execution on smaller graph sizes rather than risking stability with further optimization attempts.

\subsection{Environment}
Experiments were run in a standard Python 3.8+ environment. We utilized \texttt{numpy} for numerical operations and sampling, \texttt{pandas} for data aggregation, and \texttt{matplotlib/seaborn} for visualization.

\clearpage

% Requirement 6: The results of your experiments
\section{Results and Analysis}

The following analysis is based on the data collected from our 540 experimental runs.

\subsection{Spanner Size Validation}
We compared the empirical spanner size $|E_S|$ against the theoretical upper bound $O(k \cdot n^{1+1/k})$.

\textbf{Analysis:}
Figure \ref{fig:spanner_size} plots the spanner size against $k$ for different graph sizes. We observe a sharp exponential decay in spanner size as $k$ increases. Crucially, the \textbf{Spanner Size Ratio} was consistently found to be below 1.0. This indicates that the constant factors in the Baswana-Sen algorithm are small; in practice, the algorithm produces spanners even sparser than the worst-case theoretical bound suggests.

\begin{figure}[H]
  \centering
  \includegraphics[width=0.9\textwidth]{figures/spanner_size_vs_k.png}
  \caption{Empirical spanner size decreases as $k$ increases. The curves follow the theoretical prediction of $n^{1+1/k}$.}
  \label{fig:spanner_size}
  \vspace{0.5em}
  \footnotesize
  \href{https://github.com/Tamircohen28/baswana-sen-spanner-experiments/blob/main/results/figures/spanner_size_vs_k.png}{\textcolor{blue}{[View source image on GitHub]}}
\end{figure}

\subsection{Stretch Quality}
We analyzed the trade-off between the stretch parameter $k$ and the actual observed distortion using our sampled metrics.

\textbf{Analysis:}
\begin{itemize}
    \item \textbf{Max Stretch:} As shown in Figure \ref{fig:stretch_quality} (left), the maximum stretch observed rarely exceeded the theoretical bound of $2k-1$. In dense graphs, the max stretch often reached the bound, validating that the bound is tight for worst-case edges.
    \item \textbf{Average Stretch:} Figure \ref{fig:stretch_quality} (right) reveals that the average stretch is significantly lower than the parameter $k$ would suggest. For example, with $k=3$ (max stretch 5), the average stretch was consistently observed to be much lower (often $< 1.5$). This implies that while the algorithm guarantees a worst-case bound, the vast majority of paths in the spanner remain very close to their original lengths.
\end{itemize}

\begin{figure}[H]
  \centering
  \includegraphics[width=0.9\textwidth]{figures/stretch_vs_k.png}
  \caption{(Left) Max stretch adheres to the $2k-1$ bound. (Right) Average stretch remains extremely low even for higher $k$.}
  \label{fig:stretch_quality}
  \vspace{0.5em}
  \footnotesize
  \href{https://github.com/Tamircohen28/baswana-sen-spanner-experiments/blob/main/results/figures/stretch_vs_k.png}{\textcolor{blue}{[View source image on GitHub]}}
\end{figure}

\subsection{Runtime and Scalability}
A key claim of the Baswana-Sen paper is linear-time complexity. We validated this by measuring runtime across our graph sizes ($n=500$ to $2000$).

\textbf{Analysis:}
Figure \ref{fig:runtime} demonstrates the runtime performance. The spanner construction time scales linearly with the number of edges, confirming the $O(m)$ complexity. The graph generation often dominates the total runtime for dense graphs, highlighting the efficiency of the spanner algorithm itself.

\begin{figure}[H]
  \centering
  \includegraphics[width=0.9\textwidth]{figures/runtime_vs_n.png}
  \caption{Runtime scaling. The linear slope of the Spanner Construction time confirms $O(m)$ complexity.}
  \label{fig:runtime}
  \vspace{0.5em}
  \footnotesize
  \href{https://github.com/Tamircohen28/baswana-sen-spanner-experiments/blob/main/results/figures/runtime_vs_n.png}{\textcolor{blue}{[View source image on GitHub]}}
\end{figure}

\clearpage

% Requirement 7: The conclusions and directions for future work
\section{Conclusions and Future Work}

\subsection{Conclusions}
This project successfully replicated the Baswana-Sen algorithm and validated its theoretical properties through rigorous experimentation on graphs up to 2000 vertices.

\begin{enumerate}
    \item \textbf{Efficiency Confirmed:} The algorithm runs in expected linear time, making it one of the most efficient methods for spanner construction available.
    \item \textbf{Sparsity vs. Distortion:} We confirmed the size bound of $O(n^{1+1/k})$. More importantly, we found that the \textbf{average distortion} is remarkably low, making these spanners highly effective for real-world applications where ``average case'' performance often matters more than worst-case guarantees.
    \item \textbf{Robustness:} The algorithm proved robust across various graph densities ($p$) and sizes ($n$), consistently producing connected, valid spanners.
\end{enumerate}

\subsection{Future Work}
Future extensions of this work could include:
\begin{itemize}
    \item \textbf{Performance Optimization:} Resolving the technical issues with multi-processing to enable parallel execution of trials, which would allow testing on significantly larger graphs ($n > 10{,}000$).
    \item \textbf{Weighted Graphs:} Extending the experiments to graphs with high variance in edge weights to verify if the stretch distribution holds.
    \item \textbf{Topology Analysis:} Testing the algorithm on non-random topologies, such as scale-free networks or planar graphs.
\end{itemize}

\clearpage

\section*{Repository and Code}

The complete source code, experiment scripts, and results are available in the project repository:

\begin{itemize}
    \item \href{https://github.com/Tamircohen28/baswana-sen-spanner-experiments}{Project Repository}
    \item \textbf{Notebooks (Google Colab):}
    \begin{itemize}
        \item \href{https://colab.research.google.com/github/Tamircohen28/baswana-sen-spanner-experiments/blob/main/notebooks/01_baswana_sen_sanity_check.ipynb}{01\_baswana\_sen\_sanity\_check.ipynb}
        \item \href{https://colab.research.google.com/github/Tamircohen28/baswana-sen-spanner-experiments/blob/main/notebooks/02_experiments_main.ipynb}{02\_experiments\_main.ipynb}
        \item \href{https://colab.research.google.com/github/Tamircohen28/baswana-sen-spanner-experiments/blob/main/notebooks/03_plots_and_results.ipynb}{03\_plots\_and\_results.ipynb}
    \end{itemize}
\end{itemize}

\section*{References}

\begin{enumerate}
    \item \textbf{Baswana, S., \& Sen, S. (2007).} "A Simple and Linear-Time Randomized Algorithm for Computing Sparse Spanners in Weighted Graphs." \textit{Random Structures \& Algorithms}, 30(4), 532--563.
    \item \textbf{Peleg, D., \& Schäffer, A. (1989).} "Graph spanners." \textit{Journal of Graph Theory}, 13(1), 99--116.
    \item \textbf{Course Materials:} Syllabus and Report Guidelines (Prof. Michael Elkin).
    \item \textbf{Project Repository:} Cohen, T. (2025). \texttt{baswana-sen-spanner-experiments}.
\end{enumerate}

\end{document}